\documentclass[journal]{IEEEtran}

% Compilar com XeLaTeX ou LuaLaTeX
\usepackage[utf8]{inputenc}
\usepackage[T1]{fontenc}
\usepackage[brazil]{babel}
\usepackage{geometry}
\usepackage{graphicx}
\usepackage{titling}
\usepackage{fancyhdr}
\usepackage{enumitem}
\usepackage{titlesec}
\usepackage{xcolor}
\usepackage{tcolorbox}
\usepackage{hyperref}
\usepackage{tocloft}
\usepackage{amsfonts}
\usepackage{tikz}
\usepackage{everypage}
\usepackage{fontspec} % Importante para fontes modernas




% Cabeçalho e rodapé
\pagestyle{fancy}
\fancyhf{}\rhead{\textit{Engenharia de sistemas}}
\lhead{\textit{System Requirements Document (SRD) / System Specification}}%alterar
\fancyfoot[C]{\thepage}

% Contador de página para ativar barras só a partir da 3ª página
\newcounter{mypage}

\AddEverypageHook{%
  \stepcounter{mypage}%
  \ifnum\themypage>1
    % Barra superior
    \begin{tikzpicture}[remember picture,overlay]
      \fill[gray!20] (current page.north west) rectangle ([yshift=-2.65cm]current page.north east);
      \node[anchor=north, text=black, font=\bfseries\large] at ([yshift=-0.9cm]current page.north)
      {};
    \end{tikzpicture}
    % Barra inferior
    \begin{tikzpicture}[remember picture,overlay]
  \fill[gray!20] (current page.south west) rectangle ([yshift=2.1cm]current page.south east);
      \node[anchor=south, text=black, font=\bfseries\small] at ([yshift=0.9cm]current page.south) {\textit{Universidade Federal de Santa Catarina }};
    \end{tikzpicture}
  \fi
}



% Estilo de sumário
\cftsetindents{section}{0em}{2em}
\cftsetindents{subsection}{2em}{3em}
\cftsetindents{subsubsection}{5em}{4em}

% Formato das seções
\titleformat{\section}{\bfseries\large}{\thesection.}{1em}{}



%-----------------------------------------------------------
%-----------------------------------------------------------
%-----------------------------------------------------------



% Capa
\begin{document}
\begin{titlepage}

    \centering
    \vspace*{3cm}
    {\Huge\bfseries System Requirements Document (SRD) / System Specification \\[1.5em]}

    {\large\textit{}}\\[7em]
    \includegraphics[width=0.3\textwidth]{ufsc}\\[1em]
    {\Large \textit{Engenharia de sistemas}}
    \vfill
    %\today
\end{titlepage}



%-----------------------------------------------------------
%-----------------------------------------------------------
%-----------------------------------------------------------



% Folha de rosto
\newpage

\begin{center}
    \Large\textbf{System Requirements Document (SRD) / System Specification}\\[2em]
    \large Contains high-level system requirements
        (functional, performance, interface, environmental,
        etc.).
        Should be traceable to stakeholder needs (e.g.,
        Customer Requirements Document). )%alterar

    %\textbf{Revisado em:} 12 de dezembro de 2023\\
    %\textbf{Organização:} RoboCore\\[6em]
    %\textbf{Link:} \href{https://events.robocore.net}{https://events.robocore.net}
\end{center}



%-----------------------------------------------------------
%-----------------------------------------------------------
%-----------------------------------------------------------



% Sumário visual personalizado
\newpage
\section*{Sumário}


\begin{tcolorbox}[colback=gray!20, colframe=white, boxrule=0pt, arc=0pt]
\textbf{1. Requisitos} \hfill \textbf{\pageref{sec:requisitos}}
\end{tcolorbox}

\begin{tcolorbox}[colback=gray!20, colframe=white, boxrule=0pt, arc=0pt]
\textbf{2. Especificação do sistema} \hfill \textbf{\pageref{sec:especificacao}}
\end{tcolorbox}



%-----------------------------------------------------------
%-----------------------------------------------------------
%-----------------------------------------------------------



\newpage
% conteudo
\input{conteudo}
\input{contributors}
\end{document}
