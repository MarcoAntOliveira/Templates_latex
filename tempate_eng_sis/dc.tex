\documentclass[journal]{IEEEtran}

% Pacotes essenciais
\usepackage[utf8]{inputenc}
\usepackage[T1]{fontenc}
\usepackage[brazil]{babel}
\usepackage{geometry}
\usepackage{graphicx}
\usepackage{titling}
\usepackage{fancyhdr}
\usepackage{enumitem}
\usepackage{titlesec}
\usepackage{xcolor}
\usepackage{tcolorbox}
\usepackage{hyperref}
\usepackage{tocloft}
\usepackage{amsfonts}
\usepackage{tikz}
\usepackage{everypage}
\usepackage{amsfonts}

% Margens
\geometry{a4paper, margin=2.5cm}

% Cabeçalho e rodapé
\pagestyle{fancy}
\fancyhf{}
\rhead{\textit{Engenharia de sistemas}}
\lhead{\textit{Verification \& Validation Plan} }
\fancyfoot[C]{\thepage}

% Contador de página para ativar barras só a partir da 3ª página
\newcounter{mypage}
\AddEverypageHook{%
  \stepcounter{mypage}%
  \ifnum\themypage>2
    % Barra superior
    \begin{tikzpicture}[remember picture,overlay]
      \fill[gray!20] (current page.north west) rectangle ([yshift=-1.8cm]current page.north east);
      \node[anchor=north, text=black, font=\bfseries\large] at ([yshift=-0.9cm]current page.north)
      {};
    \end{tikzpicture}
    % Barra inferior
    \begin{tikzpicture}[remember picture,overlay]
      \fill[gray!20] (current page.south west) rectangle ([yshift=2.0cm]current page.south east);
      \node[anchor=south, text=black, font=\bfseries\small] at ([yshift=0.9cm]current page.south) {\textit{Universidade Federal de Santa Catarina} };
    \end{tikzpicture}
  \fi
}


% Estilo de sumário
\cftsetindents{section}{0em}{2em}
\cftsetindents{subsection}{2em}{3em}
\cftsetindents{subsubsection}{5em}{4em}

% Formato das seções
\titleformat{\section}{\bfseries\large}{\thesection.}{1em}{}

% Capa
\begin{document}
\begin{titlepage}
    \centering
    \vspace*{3cm}
    {\Huge\bfseries Verification \& Validation Plan \\[1.5em]}

    {\large\textit{}}\\[6em]
    \includegraphics[width=0.3\textwidth]{ufsc}\\[2em]
    {\Large \textit{Engenharia de sistemas}}
    \vfill
    %\today
\end{titlepage}

% Folha de rosto
\newpage
\thispagestyle{empty}
\begin{center}
    \Large\textbf{Verification \& Validation Plan}\\[2em]
    \large
    \begin{itemize}
        \item \textit{Outlines how requirements will be verified
        (testing, analysis, inspection,
        demonstration).}
        \item \textit{Includes Acceptance Criteria and TestConcept\\[4em]}
    \end{itemize}

    %\textbf{Revisado em:} 12 de dezembro de 2023\\
    %\textbf{Organização:} RoboCore\\[6em]
    %\textbf{Link:} \href{https://events.robocore.net}{https://events.robocore.net}
\end{center}

% Sumário visual personalizado
\newpage
\section*{Sumário}
\renewcommand{\contentsname}{}

\begin{tcolorbox}[colback=gray!20, colframe=white, boxrule=0pt, arc=0pt]
\textbf{1. \textit{Introdução}} \hfill \textbf{03}
\end{tcolorbox}
\begin{tcolorbox}[colback=gray!20, colframe=white, boxrule=0pt, arc=0pt]
\textbf{2. A Competição} \hfill \textbf{03}
\end{tcolorbox}
\begin{tcolorbox}[colback=gray!20, colframe=white, boxrule=0pt, arc=0pt]
\textbf{3. Especificações dos Robôs} \hfill \textbf{03}
\end{tcolorbox}
\begin{tcolorbox}[colback=gray!20, colframe=white, boxrule=0pt, arc=0pt]
\textbf{4. O Percurso} \hfill \textbf{04}
\end{tcolorbox}
\begin{tcolorbox}[colback=gray!20, colframe=white, boxrule=0pt, arc=0pt]
\textbf{5. Desenvolvimento da Competição} \hfill \textbf{05}
\end{tcolorbox}
\begin{tcolorbox}[colback=gray!20, colframe=white, boxrule=0pt, arc=0pt]
\textbf{6. Regras para as Disputas} \hfill \textbf{06}
\end{tcolorbox}
\begin{tcolorbox}[colback=gray!20, colframe=white, boxrule=0pt, arc=0pt]
\textbf{7. Penalidades} \hfill \textbf{07}
\end{tcolorbox}
\begin{tcolorbox}[colback=gray!20, colframe=white, boxrule=0pt, arc=0pt]
\textbf{8. Módulo de Início} \hfill \textbf{08}
\end{tcolorbox}

\newpage
\begin{titlepage}
\begin{center}
    \Large\textbf{Contributors}\\[2em]

    \begin{itemize}
        \item \textbf{Marcos Antonio Tomé Oliveira} \\\textit{Graduando em engenharia mecatrônica}\\[0.5em]

    \end{itemize}

    %\textbf{Revisado em:} 12 de dezembro de 2023\\
    %\textbf{Organização:} RoboCore\\[6em]
    %\textbf{Link:} \href{https://events.robocore.net}{https://events.robocore.net}
\end{center}
\end{titlepage}
% Conteúdo das seções
\newpage
\section{\textit{Introdução}}

\subsection{Sistemas estáticos x sistemas dinâmicos}
\begin{itemize}
    \item \textbf{Sistemas Estáticos:} saída depende apenas do valor atual da entrada.
    \begin{itemize}
        \item Não possui memória
        \item Sem dependência temporal
        \item Representado por função algébrica
    \end{itemize}
    \item \textbf{Sistemas Dinâmicos:} saída depende de entradas passadas.
    \begin{itemize}
        \item Possui memória
        \item Representado por equações diferenciais ou de diferenças
    \end{itemize}
\end{itemize}

\section{Controlador PID}

\subsection{Ganho proporcional}

\textbf{Efeitos Positivos}
\begin{itemize}
    \item Reduz tempo de subida
    \item Atua na velocidade e precisão
    \item Melhora rejeição a distúrbios
\end{itemize}

\textbf{Efeitos Negativos}
\begin{itemize}
    \item Overshoot excessivo
    \item Não elimina erro em regime
    \item Pode causar instabilidade
\end{itemize}

\subsection{Efeitos do Ganho Integral}
\textbf{Efeitos Positivos}
\begin{itemize}
    \item Elimina erro em regime permanente
    \item Aumenta robustez
    \item Considera erros acumulados
\end{itemize}

\textbf{Efeitos Negativos}
\begin{itemize}
    \item Pode causar oscilações
    \item Aumenta tempo de resposta
    \item Instabilidade com ganho elevado
\end{itemize}

\subsection{Ganho Derivativo}

O ganho derivativo prevê o comportamento futuro do erro com base na sua variação.

\textbf{Como funciona:}
\begin{itemize}
    \item Calcula a derivada do erro
    \item Aplica ganho proporcional à taxa de variação
\end{itemize}

\textbf{Efeitos:}
\begin{itemize}
    \item Amortece oscilações
    \item Reduz overshoot
    \item Melhora estabilidade
    \item Não atua em erro constante
\end{itemize}

\end{document}
