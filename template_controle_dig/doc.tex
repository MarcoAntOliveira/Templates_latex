\documentclass[a4]{article}

% Pacotes essenciais
\usepackage[utf8]{inputenc}
\usepackage[T1]{fontenc}
\usepackage[brazil]{babel}
\usepackage{geometry}
\usepackage{graphicx}
\usepackage{titling}
\usepackage{fancyhdr}
\usepackage{enumitem}
\usepackage{titlesec}
\usepackage{xcolor}
\usepackage{tcolorbox}
\usepackage{hyperref}
\usepackage{tocloft}
\usepackage{amsfonts}
\usepackage{tikz}
\usepackage{everypage}
\usepackage{amsfonts}
\usepackage{amsmath}
\usepackage{xcolor}
\usepackage{etoolbox} % Necessário para patching

\usepackage{tikz}
\usetikzlibrary{arrows.meta, positioning}



% Redefine o \subsection para adicionar uma linha horizontal
\preto{\subsection}{\vspace{1em}\noindent\textcolor{cyan!70!black}{\rule{\linewidth}{0.5pt}}\par\vspace{0.5em}}

\preto{\subsubsection}{\vspace{1em}\noindent\textcolor{cyan!70!black}{\rule{\linewidth}{0.5pt}}\par\vspace{0.5em}}

\definecolor{MidnightBlue}{RGB}{25,25,112} % ou a cor exata usada na sua barra
\definecolor{customblue}{RGB}{0, 0, 255}
\definecolor{customgreen}{RGB}{0,190, 255}
% Margens
\geometry{a4paper, margin=2.5cm}

% Cabeçalho e rodapé
\pagestyle{fancy}
\fancyhf{}\rhead{\textit{\textcolor{customblue}{Controle Moderno}}}
%\lhead{\textit{\textcolor{customblue}{Controle digital de aeronaves}}}%alterar
\fancyfoot[C]{\thepage}
\fancyhead[L]{\textit{\textcolor{customblue}{\nouppercase{\leftmark}}}}

% Contador de página para ativar barras só a partir da 3ª página
\newcounter{mypage}

\AddEverypageHook{%
  \stepcounter{mypage}%
  \ifnum\themypage>1
    % Barra superior
    \begin{tikzpicture}[remember picture,overlay]
      \fill[cyan!15] (current page.north west) rectangle ([yshift=-1.8cm]current page.north east);
      \node[anchor=north, text=black, font=\bfseries\large] at ([yshift=-0.9cm]current page.north)
      {};
    \end{tikzpicture}
    % Barra inferior
    \begin{tikzpicture}[remember picture,overlay]
      \fill[cyan!15] (current page.south west) rectangle ([yshift=2.0cm]current page.south east);
      \node[anchor=south, text=black, font=\bfseries\small] at ([yshift=0.9cm]current page.south) {\textit{\textcolor{customblue}{Universidade Federal de Santa Catarina }}};
    \end{tikzpicture}
  \fi
}
% Formato das seções
\titleformat{\section}{\bfseries\large}{\thesection.}{1em}{}

\usepackage{tcolorbox}
\tcbuselibrary{listings, breakable}

\newtcolorbox{MathBox}{
  colback=cyan!10,
  colframe=cyan!50!black,
  boxrule=0.5pt,
  arc=4pt,
  left=4pt,
  right=4pt,
  top=4pt,
  bottom=4pt
}

%Estilização das seções e subseções
\titleformat{\section}
  {\color{MidnightBlue}\normalfont\Large\bfseries}
  {\thesection}{1em}{}
\titleformat{\subsection}
  {\color{customblue}\normalfont\large\bfseries}
  {\thesubsection}{1em}{}
\titleformat{\subsubsection}
  {\color{customblue}\normalfont\large\bfseries}
  {\thesubsection}{1em}{}


\let\oldtextbf\textbf
\renewcommand{\textbf}[1]{\textcolor{customgreen}{\oldtextbf{#1}}}

% Estilização das listas
\setlist[itemize]{label=\textcolor{customblue}{\textbullet}}
\setlist[enumerate]{label=\textcolor{customblue}{\Roman*.}}

% Estilo de sumário
\cftsetindents{section}{0em}{2em}
\cftsetindents{subsection}{2em}{3em}
\cftsetindents{subsubsection}{5em}{4em}


% Capa
\begin{document}
\begin{titlepage}

    \centering
    \vspace*{3cm}
    {\Huge\bfseries Controle Moderno \\[1.5em]}

    {\large\textit{}}\\[7em]
    \includegraphics[width=0.3\textwidth]{images/ufsc}\\[1em]
    {\Large \textit{Controle moderno}}
    \vfill
    %\today
\end{titlepage}

% Folha de rosto
\newpage
\section*{Contributors}
\begin{center}
    \Large\textbf{Controle Moderno}\\[2em]
    \large Lists applicable industry, regulatory, and safety standards (e.g., ISO, SAE, MIL-
     STD, DO-178C, EASA, and so on )%alterar

    %\textbf{Revisado em:} 12 de dezembro de 2023\\
    %\textbf{Organização:} RoboCore\\[6em]
    %\textbf{Link:} \href{https://events.robocore.net}{https://events.robocore.net}
\end{center}

% Sumário visual personalizado
\newpage
\section*{Sumário}
\renewcommand{\contentsname}{}

\begin{tcolorbox}[colback=cyan!15, colframe=white, boxrule=0pt, arc=0pt]
\textbf{\textcolor{MidnightBlue}{1. Introdução}} \hfill \textbf{03}
\end{tcolorbox}

\begin{tcolorbox}[colback=cyan!15, colframe=white, boxrule=0pt, arc=0pt]
\textbf{\textcolor{MidnightBlue}{2. Controlador PID}} \hfill \textbf{\pageref{sec:pid}}
\end{tcolorbox}

\begin{tcolorbox}[colback=cyan!05, colframe=white, boxrule=0pt, arc=0pt, left=5mm]
\textcolor{MidnightBlue}{2.1 Ganho Proporcional} \hfill \textbf{\pageref{subsec:ganho_proporcional}}
\end{tcolorbox}


\begin{tcolorbox}[colback=cyan!05, colframe=white, boxrule=0pt, arc=0pt, left=5mm]
\textcolor{MidnightBlue}{2.2 Ganho Integral} \hfill \textbf{\pageref{subsec:ganho_integral}}
\end{tcolorbox}

\begin{tcolorbox}[colback=cyan!05, colframe=white, boxrule=0pt, arc=0pt, left=5mm]
\textcolor{MidnightBlue}{2.3 Ganho Derivativo} \hfill \textbf{\pageref{subsec:ganho_deriavtivo}}
\end{tcolorbox}

\begin{tcolorbox}[colback=cyan!05, colframe=white, boxrule=0pt, arc=0pt, left=5mm]
\textcolor{MidnightBlue}{2.4 Controlador PID - Implementação prática} \hfill \textbf{\pageref{subsec:implementação_prática}}
\end{tcolorbox}

\begin{tcolorbox}[colback=cyan!15, colframe=white, boxrule=0pt, arc=0pt]
\textbf{\textcolor{MidnightBlue}{3. Variantes do Controlador PID}} \hfill \textbf{\pageref{sec:variantes}}
\end{tcolorbox}

\begin{tcolorbox}[colback=cyan!15, colframe=white, boxrule=0pt, arc=0pt]
\textbf{\textcolor{MidnightBlue}{4. Controlador em Cascata}} \hfill \textbf{\pageref{sec:cascata}}
\end{tcolorbox}
\begin{tcolorbox}[colback=cyan!05, colframe=white, boxrule=0pt, arc=0pt, left=5mm]
\textcolor{MidnightBlue}{4.1 Conceitos} \hfill \textbf{\pageref{subsec:conceitos}}
\end{tcolorbox}
\begin{tcolorbox}[colback=cyan!05, colframe=white, boxrule=0pt, arc=0pt, left=5mm]
\textcolor{MidnightBlue}{4.2 criterios de Implementação} \hfill \textbf{\pageref{subsec:criterios_implemetacao}}
\end{tcolorbox}

\begin{tcolorbox}[colback=cyan!05, colframe=white, boxrule=0pt, arc=0pt, left=5mm]
\textcolor{MidnightBlue}{4.3 Sintonia} \hfill \textbf{\pageref{subsec:sintonia_cascata}}
\end{tcolorbox}

\begin{tcolorbox}[colback=cyan!15, colframe=white, boxrule=0pt, arc=0pt]
\textbf{\textcolor{MidnightBlue}{5. Represesentação em espaço de Estados  }} \hfill \textbf{\pageref{sec:espaco_estados}}
\end{tcolorbox}

\begin{tcolorbox}[colback=cyan!15, colframe=white, boxrule=0pt, arc=0pt]
\textbf{\textcolor{MidnightBlue}{6. Discretização do Espaço de Estados }} \hfill \textbf{\pageref{sec:discretizacao_espaco_estados}}
\end{tcolorbox}
\begin{tcolorbox}[colback=cyan!15, colframe=white, boxrule=0pt, arc=0pt]
\textbf{\textcolor{MidnightBlue}{7. Solução das Equações de Estado  - caso contínuo}} \hfill \textbf{\pageref{sec:solucao_espaco_estados}}
\end{tcolorbox}

\begin{tcolorbox}[colback=cyan!15, colframe=white, boxrule=0pt, arc=0pt]
\textbf{\textcolor{MidnightBlue}{8. Espaço de Estados e Função de Transferência}} \hfill \textbf{\pageref{sec:solucao_espaco_estados}}
\end{tcolorbox}
\newpage
\begin{center}
    \Large\textbf{Contributors}\\[2em]

    \begin{itemize}
        \item \textbf{Marcos Antonio Tomé Oliveira} \\\textit{Graduando em engenharia mecatrônica}\\[0.5em]

    \end{itemize}

    %\textbf{Revisado em:} 12 de dezembro de 2023\\
    %\textbf{Organização:} RoboCore\\[6em]
    %\textbf{Link:} \href{https://events.robocore.net}{https://events.robocore.net}
\end{center}
\newpage

\input{topics/introducao}
\input{topics/pid}
\input{topics/pid_variantes}
\input{topics/controle_cascata}
\input{topics/espaco_estados}
\input{topics/funcao_transferencia}
\end{document}
